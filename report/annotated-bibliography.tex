\documentclass[12pt, letterpaper]{article}
\usepackage[top=1in, bottom=1in, left=1.5in, right=1.5in]{geometry}
\usepackage{natbib}

\begin{document}
\title{Annotated Bibliography}
\author{Reginald Edwards\footnote{Ross School of Business, University of 
Michigan (\texttt{reggie@umich.edu}). Preliminary and incomplete. Please do not cite or circulate 
without permission.}}
\date{This Draft: April 2016}

\maketitle

\section{Institutional Investor Ownership}
\subsection{Drivers of Institutional Investor Ownership}
\begin{itemize}
\item \cite{lehavysloan2008}
    \begin{itemize}
    \item \textbf{Research Question.} They test Merton's (1987) model that predicts: (1) security 
value is increasing in investor recognition, (2) expected returns are decreasing in investor 
recognition, (3) the previous two relationships are increasing in the security's idiosyncratic 
risk, and (4) financing and investing activities in the underlying firm are increasing in investor 
recognition.
    
    \item \textbf{Empirical Design.} Their sample period is 1982 - 2004. They proxy for investor 
recognition by institutional investors (those filing 13F filings with the SEC). They construct 
$\Delta BREADTH$ measure as in Chen, et al. (2002). Breadth of ownership and stock returns. 
\emph{Journal of Financial Economics}.
    
    \item \textbf{Results.} Contemporaneous returns are increasing in investor recognition (Table 
3). The explanatory power of investor recognition for contemporaneous returns is incremental to 
proxies for cash flow news (Table 4). The positive relationship between investor recognition and 
firm value is stronger for stocks with higher idiosyncratic risk (Table 4). Investor recognition is 
associated with future returns (Table 5). Investor recognition is positively associated with 
contemporaneous and future investing and financing activities (Table 7).
    
    \item \textbf{Conclusions.} Investor recognition may be more important than accounting 
information in explaining the variation in stock returns. Investor recognition is an important 
determinant of expected return news and changes in investor recognition may be as important as 
earnings surprises in explaining security returns.
    
    \item \textbf{Notes.} The investor recognition proxy is fundamentally institutional investor 
ownership. Thus, the results can be interepreted as effects of increasing institutional investor 
ownership without any implications for Merton's model. They motivate the analysis as a direct test 
of Merton's model, which would be a really strong theoretical motivation, but the proxy is a poor
one. It would be less theoretically grounded, but more honest, to provide the empirical analysis and
give as one potential explanation that their breadth measure may be associated with investor 
recognition. 

The biggest open question is what determines investor recognition/institutional investor ownership?
    \end{itemize}
\item \cite{busheenoe2000}
   \begin{itemize} 
    \item \textbf{Research Question.} Does higher quality disclosure cause institutional investors to buy more of firms' shares, which then leads to higher stock price volatility for these firms? Does the type of institution, based on the length of their investment horizon, moderate or mediate these effects?

    \item \textbf{Background and Motivation} \cite{langlundholm1993} document a positive association between disclosure quality and stock return volatility. \cite{healyetal1999} document a positive association between disclosure quality and institutional investor ownership. \cite{sias1996} and \cite{potter1992} document a positive association between institutional investor ownership of firms' shares and those firms' stock return volatility. The authors propose that these findings are linked--higher disclosure quality attracts institutional investors who then cause higher stock return volatility.

    \item \textbf{Empirical Design.} Disclosure quality is measured by AIMR ratings. Institutional investors are classified as one of either transient, quasi-indexer, or dedicated institutions as in \cite{bushee1998}.

    \item \textbf{Results.} Transient institutions invest more in heavily in firms with higher disclosre rankings and add to their holdings in response to increases in disclosure rankings. Quasi-indexer institutions invest more heavily in firms with higher disclosure rankings, decrease holdings in firms that experience decreases in disclosure rankings, but do not immediately increase holdings in firms that experience increases in disclosure rankings. Dedicated institutions show no sensitivity to disclosure rating levels or changes.

    \item \textbf{Conclusions.} No attempt at identification.

    \item \textbf{Notes.} 
    \end{itemize}

\end{itemize}

\subsection{Outcomes Associated with Institutional Investor Ownership}
\begin{itemize}
\item \cite{}
   \begin{itemize}
    \item \textbf{Research Question.}
    \item \textbf{Empirical Design.}
    \item \textbf{Results.}
    \item \textbf{Conclusions.}
    \item \textbf{Notes.} 
    \end{itemize}

\item \cite{fangetal2015}
   \begin{itemize}
    \item \textbf{Research Question.} To what extent do US institutional investors directly affect 
the convergence of reporting practices (i.e., ...? ) in foreign firms to the reporting practices of U.S. firms?

\item \textbf{Background and Motivation} The authors claim that over the past three decades there has been a convergence in the accounting standards of many developing countries to that of developed countries. They argue that this is unlikely to be due to changes in mandatory disclosure regulations in each of these countries. They argue that investment by U.S. institutional investors in the firms in these developing countries is the driver of this increase in convergence.

    \item \textbf{Empirical Design.} Firms from 18 emerging markets and 20 developed markets from 
1998 to 2009. Comparability constructed as in \cite{defrancoetal2011}. $COMP\_US_{ijt}$ measured as 
the comparability of a non-US firm $i$ and US firm $j$ in a year $t$. $COMP\_US_{it}$ is this 
measure averaged over all $j$ firms. Institutional ownership is measured as fraction of shares held 
by mutual funds. Regressions of the form $\Delta COMP\_US_{t, t+1} = \alpha + \beta US\_OWN + \gamma 
X^{\prime} + \varepsilon$.

The authors' identification strategy uses the 2003 JGTRRA Act to construct an IV.

    \item \textbf{Results.} Among developed market firms, US institutional ownership in one year $t$ 
is not associated with a change in comparability to US firms from that year to the next (time $t$ to 
$t+1$). But, among emerging market firms, US institutional ownership is positively associated with 
an increase in comparability (Table 3).

    \item \textbf{Conclusions.}

    \item \textbf{Notes.} Only explore one potential mechanism of action--U.S. institutions cause firms in the developing markets to switch to a Big Four auditor (the ``auditor channel'').
    \end{itemize}

\item \cite{defondetal2011}
   \begin{itemize}
    \item \textbf{Research Question.} Did the EU's mandatory adoption of IFRS in 2005 result in 
improved comparability and did this improvement in comparability lead to increased investment by foreign mutual funds? How did the credibility of implementation of these accounting standards mediate or moderate these effects? How did the increase in uniformity of accounting policies that resulted from the adoption of IFRS mediate or moderate these effects? What are the comparable effects for domestic mutual fund ownership?

    \item \textbf{Background and Motivation} The authors quote Frits Bolkestein, the European Commissioner for the Internal Market: ``investors and other stakeholders will be able to compare like with like'' after the adoption of IFRS in the EU. The authors claim that this view--that IFRS adoption will improve comparability among member firms--is common among proponents of IFRS adoption in the EU. They then claim that the view that higher comparability will lead to higher cross-border investment is common among academics and investment professionals. The motivation for the authors' research comes from linking these two views.

    \item \textbf{Empirical Design.} 14 EU countries over 2003-2007. Measure institutional investor 
ownership as fraction of shares held by ``foreign'' mutual funds--mutual funds based outside of the focal country for each of the 14 countries of interest. Implementation credibility is proxied by the aggregate earnings quality score from \cite{leuzetal2003}. Uniformity is measured as the percentage of industry peers adopting the same GAAP.

    \item \textbf{Results.} Mandatory IFRS adoption increased ownership by foreing mutual funds when the adoption was (a) highly credible; \emph{and} (b) led to a large increase in uniformity. Mandatory IFRS adoption \emph{did not increase} domestic mutual fund ownership. IFRS adoption resulted in a greater increase in foreign investment only in countries with strong implementation credibility that \emph{also} experienced relatively large increases in uniformity. 

    \item \textbf{Conclusions.}

    \item \textbf{Notes.} Neither foreign not domestic mutual fund ownership in countries that did not adopt IFRS (e.g. the U.S.) are explicitly considered.
    \end{itemize}
    
    
\end{itemize}

\section{Financial Statement Comparability}
\subsection{Determinants of Financial Statement Comparability}
\begin{itemize}
\item  \cite{yipyoung2012}
   \begin{itemize}
    \item \textbf{Research Question.} What is the effect of IFRS adoption on cross-country 
information comparability? Does IFRS adoption differentially affect the similarity facets and difference facets of comparability? Is the improvement in the similarity facet of cross-country comparability affected by firms’ institutional environment--the common law versus code law origin of the legal system in its home country?

    \item{\textbf{Background and Motivation.}} The authors quote the IASB's assertion that ``an overemphasis on uniformity may reduce comparability by making unlike things look alike''. Taking this view, the authors identify two components of comparability. the first is the ``similiarity facet''--the degree to which similar economic events translate to similar accounting numbers. The second is the ``difference facet''--the extent to which different economic events translate to different accounting numbers. The authors claim that one stated benefit of IFRS adoption is that the information produced by firms in the countries that adopt IFRS will be more comparable to other firms in other countries that also adopt IFRS (``cross-country information comparability''). The authors claim that prior empirical evidence that attempts to answer this question has several weaknesses, including the failure to estimate the differential effects of IFRS adoption on the similarity and difference facets of comparability.

    \item \textbf{Empirical Design.}
    \item \textbf{Results.}
    \item \textbf{Conclusions.}
    \item \textbf{Notes.} 
    \end{itemize}

\end{itemize}
\subsection{Consquences of Financial Statement Comparability}
\begin{itemize}
\item \cite{kimetal2016}
    \begin{itemize}
    \item \textbf{Research Question.} What is the impact of financial statement comparability on 
\emph{ex ante} crash risk?

    \item \textbf{Empirical Design.} Their sample is from 1996 to 2013. They proxy for \emph{ex 
ante} crash risk with the implied volatility smirk of a stock's option. They define financial 
statement comparability as in \cite{defrancoetal2011}.

    \item \textbf{Results.} Enhanced comparability is associated with a less steep implied 
volatility skew (Table 2). 
    
The negative relationship between comparability and expected crash risk is stronger for firms 
with lower quality information environments, using the probability of informed trade (PIN) as a 
proxy for the quality of the information environment (Table 5-A).
    
The negative relationship between comparability and expected crash risk is stronger for firms 
with weaker external monitoring, as proxied by a composite measure based on the percentage of 
institutional holdings and analyst following (Table 5-B).
    
The negative relationship between comparability and expected crash risk is stronger for firms 
with lower product market competition, proxied by HHI (Table 5-C).

Firms with greater financial statement comparability are less likely to delay the disclosure of bad 
news, consistent with the view that comparability constrains managers' incentives and ability to 
hoard bad news.

    \item \textbf{Conclusions.} Financial statement comparability discourages managers from 
delaying the disclosure of bad news until it accumulates, which reduces investors' perceptions of a 
future stock price crash.
    
    \item \textbf{Notes.} 
    \end{itemize}

\end{itemize}
\begin{itemize}


\item \cite{bhojrajlee2002}
   \begin{itemize}
    \item \textbf{Research Question.}
    
    \item \textbf{Empirical Design.}
    
    \item \textbf{Results.}

    \item \textbf{Conclusions.}
    
    \item \textbf{Notes.} 
    \end{itemize}
\item
   \begin{itemize}
    \item \textbf{Research Question.}
    
    \item \textbf{Empirical Design.}
    
    \item \textbf{Results.}

    \item \textbf{Conclusions.}
    
    \item \textbf{Notes.} 
    \end{itemize}
\end{itemize}
\section{Impact of Institutional Investor Ownership on Financial Statement Comparability}
\begin{itemize}
\item 
\end{itemize}
\section{Impact of Financial Statement Comparability on Institutional Investor Ownership}
\begin{itemize}
\item 
\end{itemize}
\newpage
\bibliographystyle{aea}
\bibliography{articles}
\end{document}